% earth_datasheet: This is a datasheet designed for an Earth dataset.
% Adapted by Charlotte Connolly and Daniel Hueholt
% See more information including instructional guide at DOI: github.com/dmhuehol/Datasheets-for-Earth-Science-Datasets
% Based on "Datasheets for Datasets" Gebru et al. 2021 https://dl.acm.org/doi/fullHtml/10.1145/3458723 for implementation
% Formatting based on the template by Olamilekan Wahab, licensed under Creative Commons 4.0

\documentclass[letterpaper, 10 pt, transmag]{IEEEtran}

\raggedbottom
\DeclareUnicodeCharacter{2713}{\checkmark}
\usepackage{graphicx}
\usepackage{lipsum}
\usepackage[dvipsnames,table,xcdraw]{xcolor}
\usepackage{fancyhdr}
\usepackage[includeheadfoot, left=0.6in, right=0.6in, bottom=0.5in, top=0.75in, headheight=27pt]{geometry}
\usepackage{enumitem,amssymb}

\graphicspath{ {images/} } % Directory for images
\newcommand{\checked}{✓}

\title{
\LARGE \textbf{Datasheet for an Earth Science Dataset}
% Change title to whatever is appropriate for your project
\\ \normalsize Released: % Date made public
\\ \footnotesize Last updated: % Record updates
}

% Replace this placeholder author block with the true author(s) of the datasheet 
\author{ Author One \\
	Affiliation \\
	\texttt{email} \\
	%% examples of more authors
	\and
	Author Two \\
	Affiliation \\
	\texttt{email} \\
	%% \and
	%% Coauthor Name \\
	%% Affiliation \\
	%% \texttt{email} \\
}

\begin{document}
%%%%%%
\maketitle
\thispagestyle{fancy}
\pagestyle{fancy}
\fancyhead[L]{\footnotesize{\texttt{BETA VERSION FOR DEVELOPMENT [v0.5] -- see instructional guide for details! Contact the template authors:} \\
\texttt{Charlotte Connolly cconn@rams.colostate.edu, Daniel Hueholt daniel.hueholt@colostate.edu}}}
%%%%%%
\section{Purpose}

\textcolor{blue}{\subsection{For what purpose was the dataset created?}}
\textcolor{gray}{\textit{Motivation: Describe the reason for the creation of the dataset (e.g., to provide insight on a knowledge gap, or to carry out some specific task).}}

\textcolor{blue}{\subsection{Who created the dataset (e.g., which individual or research group), on behalf of which entity (e.g., institution or company), and under what funding (e.g., grantor[s] and grant number[s])?}}
\textcolor{gray}{\textit{Motivation: Provide clarity about the authorship and funding source of the dataset.}}

\textcolor{blue}{\subsection{Was the author of the datasheet involved in creating the dataset? If not, please describe their relation to the dataset.}}
\textcolor{gray}{\textit{Motivation: Document the authorship of the datasheet, which may be different than the creator of the dataset.}}

\textcolor{blue}{\subsection{What tasks has the dataset been used for? Please provide a description and/or citation(s); if there is a repository that archives uses of the dataset, provide a link.}}
\textcolor{gray}{\textit{Motivation: Document use cases of the dataset.}}

\textcolor{blue}{\subsection{Any other comments?}}
\textcolor{gray}{\textit{Motivation: Space for any other relevant information about the creation of the dataset.}}  

\vspace{10mm}

\section{Structure and Processing}
This section concerns technical aspects of the dataset. If documented elsewhere, provide a brief description and stable link (permanent reference, e.g., a DOI) in the relevant question(s).

\textcolor{blue}{\subsection{What type(s) of data is/are contained in this dataset? (e.g., model output, observational data, reanalysis, etc.)}}
\textcolor{gray}{\textit{Motivation: Basic information about data classification.}}

\textcolor{blue}{\subsection{What is the data? (e.g., file format, dimensionality, variables, metadata, spatiotemporal coverage). Is there important metadata in the data filenames? If so, document this here.}}
\textcolor{gray}{\textit{Motivation: Provide format and characteristics of the data.}}

\textcolor{blue}{\subsection{Is this dataset derived from a preexisting dataset? (e.g., variable[s] drawn from a modeling experiment). If so, please describe the process or link to the relevant paper.}} 
\textcolor{gray}{\textit{Motivation: Describe whether a dataset is drawn or derived from a preexisting dataset.}}

\textcolor{blue}{\subsection{What processing, if any, has been applied to this data? Is any code used to process the data available? If so, please provide a stable link or other access point.}}
\textcolor{gray}{\textit{Motivation: Minimal description of the process to obtain the data described by this datasheet from its unprocessed form.}}

\textcolor{blue}{\subsection{Is any unprocessed data available? If so, please provide a stable link.}}
\textcolor{gray}{\textit{Motivation: Clarify the location of unprocessed data to facilitate reproducibility or unforeseen future uses, if possible.}}

\textcolor{blue}{\subsection{Is any relevant information known to be missing from the dataset? If so, please provide an explanation.}}
\textcolor{gray}{\textit{Motivation: Document data missing or lost from the dataset.}}

\textcolor{blue}{\subsection{Are there known sources of noise, redundancies, or errors in the dataset? If so, please provide a description.}}
\textcolor{gray}{\textit{Motivation: Provide information about relevant known technical issues that affect all or portions of the dataset.}}

\textcolor{blue}{\subsection{Is the dataset self-contained? If external resources are involved, please describe them and any associated restrictions.}}
\textcolor{gray}{\textit{Motivation: Explicitly track external dependencies that may otherwise go unacknowledged.}}

\textcolor{blue}{\subsection{Any other comments?}}
\textcolor{gray}{\textit{Motivation: Space for any other relevant information about the structure and processing of the dataset.}}  
\vspace{10mm}

\section{Distribution and Maintenance}

\textcolor{blue}{\subsection{Is the dataset available to others? If not, why? If so, how will it be distributed (e.g., FTP, Earth System Grid, personal communication)? Is there a stable link?}}
\textcolor{gray}{\textit{Motivation: Document availability and access to the dataset.}}

\textcolor{blue}{\subsection{Who is/are the point(s) of contact for this dataset?}}
\textcolor{gray}{\textit{Motivation: Provide information about who is responsible for responding to inquiries about the dataset.}}

\textcolor{blue}{\subsection{Will the dataset be updated in the future (e.g., to add new data)? If so, will older versions continue to be available?}}
\textcolor{gray}{\textit{Motivation: Clarify whether this version of the data is final.}}

\textcolor{blue}{\subsection{What license or other terms of use apply to the dataset? Please link to any relevant licensing terms or terms of use (if in the public domain, simply state this).}}
\textcolor{gray}{\textit{Motivation: Provide information about what future uses of the data are permitted.}}

\textcolor{blue}{\subsection{Is there a document that describes an important error in this dataset (e.g., an erratum)? If so, please provide a link or other access point.}}
\textcolor{gray}{\textit{Motivation: Document any corrections to the dataset.}}

\textcolor{blue}{\subsection{Who is hosting the datasheet? Will the datasheet be updated in the future?}}
\textcolor{gray}{\textit{Motivation: Document stable access to the datasheet.}}

\textcolor{blue}{\subsection{Any other comments?}}
\textcolor{gray}{\textit{Motivation: Space for any other relevant information about data distribution and maintenance.}}
\vspace{10mm}

\section{Data-dependent questions}
Responses in this section will depend on the type(s) of data within the dataset. Questions that do not apply can be left blank.


\textcolor{blue}{\subsection{How was the data generated or collected (e.g., model runs, reanalysis processes, observational measurements)? Please provide relevant citation(s); if none exist, describe why.}}
\textcolor{gray}{\textit{Motivation: Establish fundamental information about the methods used to generate or collect data in the dataset.}}

\textcolor{blue}{\subsection{Has the data been assessed against some baseline(s) (e.g., an observational product or physical laws)? If so, describe how, and provide any relevant citations.}}
\textcolor{gray}{\textit{Motivation: Document evaluation of the data within the scope of this datasheet.}}

\textcolor{blue}{\subsection{Has uncertainty quantification been carried out for this dataset? If so, describe how and provide citation(s).}}
\textcolor{gray}{\textit{Motivation: Provide information about known uncertainties.}}

\textcolor{blue}{\subsection{Did the method of generation or collection of the data change within the scope of the dataset?}}
\textcolor{gray}{\textit{Motivation: Describe important changes to instruments or methodology within the dataset.}}

\textcolor{blue}{\subsection{Are there any unexplained but relevant numerical values (``magic numbers") that go into the data generation, collection, or processing? (e.g., calibration constants, hyperparameters)}}
\textcolor{gray}{\textit{Motivation: Define unique numerical values that exist within or impact this data, but may not be documented elsewhere.}}

\textcolor{blue}{\subsection{Was a model used to generate data in this dataset? If so, please describe the  configuration and any modifications.}}
\textcolor{gray}{\textit{Motivation: Record the exact model setup used to create data.}}

\textcolor{blue}{\subsection{Is this dataset an ensemble? If so, how many members are there? Are there any differences between members? Describe the perturbation of the members, and any relevant sampling limitations (e.g., ocean states).}}
\textcolor{gray}{\textit{Motivation: Describe the sampling, construction, and any important limitations of the ensemble.}}

\textcolor{blue}{\subsection{Are there relevant categories, groupings, or labels within the data? If so, how are these determined?}}
\textcolor{gray}{\textit{Motivation: Document group definitions within the data.}}

\textcolor{blue}{\subsection{Can users contribute to this dataset (e.g., citizen science or human labeling)? If so, please describe the process including evaluation or verification.}}
\textcolor{gray}{\textit{Motivation: Describe if the data includes user contributions.}}

\textcolor{blue}{\subsection{Are there specific tasks for which the dataset should not be used? If so,
please provide a description.}}
\textcolor{gray}{\textit{Motivation: Address relevant gaps or inadequacies of the data for specific use cases.}}

\textcolor{blue}{\subsection{What are the direct or downstream impacts on humans from this dataset? The non-comprehensive checklist below is intended to prompt the author to think of common impacts from data. Please check all that apply, and include a brief text description with stable links to any references. Additionally, please document potential impacts relevant to the scope of the dataset that are not included on the checklist.}}
\textcolor{gray}{\textit{Motivation: Reflect on potential impacts (direct or downstream) of the dataset.}}

\newlist{todolist}{itemize}{2}
\setlist[todolist]{label=$\square$}

Direct
\begin{todolist}
% Check boxes by changing \item to \item[\checked]
  \item Does this dataset support reproducibility of a specific scientific finding or figure?
  \item Were there notable CO\textsubscript{2} emissions in creating this dataset? \\ 
  (e.g., from large machine learning models)
  \item Were there notable land use impacts from equipment? \\
  (e.g., in situ instruments during a field experiment)
  \item Was this dataset created through co-production of research? \\
  (e.g., for fieldwork in vulnerable communities)
  \item Does this dataset include identifying information? \\
  (e.g., community-level data, social information) 
\end{todolist}

Downstream
\begin{todolist}
  \item Is this dataset intended for development of a research tool? \\
  (e.g., model improvement, sensor design)
  \item Does this dataset support further use for novel research? (e.g., unrelated scientific studies)
  \item Would analysis of this dataset be policy relevant? \\
  (e.g., climate, environmental, public health issues)
  \item Would this dataset be considered actionable science? \\ 
  (e.g., completed with use by a specific stakeholder in mind)
  \item Could this dataset inspire behavioral changes? \\
  (e.g., change agricultural practices, city planning)
  \item Could this dataset affect operational forecasting? \\
  (e.g., improve models, forecasting, predictability)
\end{todolist}

\textcolor{blue}{\subsection{What biases were present in the construction or use of the dataset? The checklist below provides a non-comprehensive list of common examples in Earth science. Please check all that apply, and include a brief text description with stable links to any references. Additionally, please document any biases within the scope of the dataset that are not included in the checklist.}}
\textcolor{gray}{\textit{Motivation: Reflect on potential biases present in the dataset.}}

\begin{todolist}
    \item Geographic bias (e.g., restricted or weighted to specific regions)
    \item Model bias (e.g., error relative to evaluation product)
    \item Sensor bias (e.g., calibration)
    \item Temporal bias (e.g., diurnal cycle, restrictions on detection)
    \item Seasonal bias (e.g., seasonal cycle)
    \item Bias towards extreme or standard conditions (e.g., catchment error, failure to represent extremes)
    \item Unbalanced sampling (e.g., unequal classes)
    \item Adversarial impacts on data (e.g., fraudulent samples in crowdsourced data)
    \item Label bias (e.g., subjective labeling)
    \item Threshold sensitivity (e.g., for an extreme index)
    \item Regime dependence (e.g., convective structure, mode of variability)
    \item Selection bias (e.g., case studies, survivorship effects, loss of historical data over time)
\end{todolist}

\textcolor{blue}{\subsection{Any other comments? Are there any other citations necessary to document some important aspect of the data? If so, provide the citation(s) and describe their purpose.}}
\textcolor{gray}{\textit{Motivation: Space for any other relevant information.}}


\bigskip
 
\bibliographystyle{unsrt}  
\bibliography{cite}
\end{document}

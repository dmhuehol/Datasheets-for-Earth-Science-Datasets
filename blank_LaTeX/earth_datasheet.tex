% earth_datasheet: This is a datasheet designed for an Earth dataset.
% Adapted by Charlotte Connolly and Daniel Hueholt
% See more information including instructional guide at DOI: github.com/dmhuehol/Datasheets-for-Earth-Science-Datasets
% Based on "Datasheets for Datasets" Gebru et al. 2021 https://dl.acm.org/doi/fullHtml/10.1145/3458723 for implementation
% Formatting based on the template by Olamilekan Wahab, licensed under Creative Commons 4.0

\documentclass[letterpaper, 10 pt, transmag]{IEEEtran}

\raggedbottom
\DeclareUnicodeCharacter{2713}{\checkmark}
\usepackage{graphicx}
\usepackage{lipsum}  
\usepackage[dvipsnames,table,xcdraw]{xcolor}
\usepackage{fancyhdr}
\usepackage[includeheadfoot, left=0.6in, right=0.6in, bottom=0.5in, top=0.75in, headheight=27pt]{geometry}
\usepackage{enumitem,amssymb}

\graphicspath{ {images/} } % Directory for images
\newcommand{\checked}{✓}

\title{
\LARGE \textbf{Datasheet for an Earth Science Dataset}
% Change title to whatever is appropriate for your project
\\ \normalsize Released: % Date made public
\\ \footnotesize Last updated: % Record updates
}

% Replace this placeholder author block with the true author(s) of the datasheet 
\author{ Author One \\
	Department Name \\
	University Name \\
	Location \\
	\texttt{email} \\
	%% examples of more authors
	\and
	Author Two \\
	Department Name \\
	University Name \\
	Location \\
	\texttt{email} \\
	%% \and
	%% Coauthor Name \\
	%% Affiliation \\
	%% Address \\
	%% \texttt{email} \\
}

\begin{document}
%%%%%%
\maketitle
\thispagestyle{fancy}
\pagestyle{fancy}
\fancyhead[L]{\footnotesize{\texttt{BETA VERSION FOR DEVELOPMENT [v0.2] -- see instructional guide for details! Contact the template authors:} \\
\texttt{Charlotte Connolly cconn@rams.colostate.edu, Daniel Hueholt daniel.hueholt@colostate.edu}}}
%%%%%%
\section{Purpose}

\textcolor{blue}{\subsection{For what purpose was the dataset created?}}
\textcolor{gray}{\textit{Motivation: Describe the reason for the creation of the dataset (e.g., to provide insight on a knowledge gap, or to carry out some specific task).}}

\textcolor{blue}{\subsection{Who created the dataset (e.g., which individual or research group), on behalf of which entity (e.g., institution or company), and under what funding (e.g., grantor[s] and grant number[s])?}}
\textcolor{gray}{\textit{Motivation: Provide clarity about the authorship and funding source of the given dataset.}}

\textcolor{blue}{\subsection{Was the author of the datasheet involved in creating the dataset? If so, how? If not, please describe your relation to this dataset.}}
\textcolor{gray}{\textit{Motivation: Document the authorship of the datasheet, which may be different than the creator of the dataset.}}

\textcolor{blue}{\subsection{What tasks has the dataset been used for? Please provide a description and/or citation(s); if there is a repository that archives uses of the dataset, provide a permanent reference (stable link, e.g., a DOI) here.}}
\textcolor{gray}{\textit{Motivation: Document use cases of the dataset within the scope of this datasheet.}}

\textcolor{blue}{\subsection{Any other comments?}}
\textcolor{gray}{\textit{Motivation: Space for any other relevant information about the creation of the dataset.}}  

\vspace{10mm}

\section{Structure and Processing}
This section concerns technical aspects of the dataset. If this information is documented elsewhere you may simply provide a brief description and stable link in the relevant question(s).

\textcolor{blue}{\subsection{What type of data is contained in this dataset? (e.g., is it model output, observational data, reanalysis, etc.?)}}
\textcolor{gray}{\textit{Motivation: Basic information about the fundamental classification of your data.}}

\textcolor{blue}{\subsection{What is the data? (e.g., file format, dimensionality, variables and metadata, spatiotemporal coverage). Is there important metadata contained in the filename of the data? If so, document this here.}}
\textcolor{gray}{\textit{Motivation: Provide format and characteristics of the data.}}

\textcolor{blue}{\subsection{What processing has been applied to this data?}}
\textcolor{gray}{\textit{Motivation: Minimal description of the process to obtain the data described by this datasheet from its unprocessed form.}}

\textcolor{blue}{\subsection{Is the unprocessed data available in addition to the processed data? If so, please provide a stable link to the unprocessed data.}}
\textcolor{gray}{\textit{Motivation: Clarify the location of the unprocessed data to facilitate reproducibility or unforeseen future uses, if possible.}}

\textcolor{blue}{\subsection{Is the code used to process the data available? If so, please provide a stable link or other access point.}}
\textcolor{gray}{\textit{Motivation: Share processing methodology to facilitate reproducibility, if possible.}}

\textcolor{blue}{\subsection{Is this dataset derived from another dataset? If so, how?}} 
\textcolor{gray}{\textit{Motivation: Describe whether a dataset is drawn or derived from a preexisting dataset (e.g., field campaign, model intercomparison).}}

\textcolor{blue}{\subsection{Is any relevant information known to be missing from the dataset? If so, please provide an explanation.}}
\textcolor{gray}{\textit{Motivation: Document missing data within the dataset.}}

\textcolor{blue}{\subsection{Are there any sources of noise, redundancies, or errors in the dataset? If so, please provide a description.}}
\textcolor{gray}{\textit{Motivation: Provide information about relevant known technical issues that affect all or portions of the dataset.}}

\textcolor{blue}{\subsection{Is the dataset self-contained, or does it rely on external resources? Please describe external resources and any associated restrictions, as well as relevant links or other access points.}}
\textcolor{gray}{\textit{Motivation: Explicitly track external dependencies that may otherwise go unacknowledged.}}

\textcolor{blue}{\subsection{Any other comments?}}
\textcolor{gray}{\textit{Motivation: Space for any other relevant information about the structure and processing of the dataset.}}  
\vspace{10mm}

\section{Distribution and Maintenance}

\textcolor{blue}{\subsection{How will the dataset be distributed (e.g., FTP server, Earth System Grid, Amazon Web Services, etc.)? Is there a DOI or other stable link?}}
\textcolor{gray}{\textit{Motivation: Document stable access to the dataset.}}

\textcolor{blue}{\subsection{Who is/are the point(s) of contact for this dataset?}}
\textcolor{gray}{\textit{Motivation: Provide information about who is responsible for responding to inquiries about this dataset.}}

\textcolor{blue}{\subsection{Is the dataset complete or will it be updated in the future (e.g., to add new data, or make corrections)? Will older versions continue to be available?}}
\textcolor{gray}{\textit{Motivation: Clarify whether this version of the data is final.}}

\textcolor{blue}{\subsection{What license or other terms of use is the dataset distributed under? Please link to any relevant licensing terms or terms of use (if in the public domain, simply state this).}}
\textcolor{gray}{\textit{Motivation: Provide information about what future uses of the data are permitted.}}

\textcolor{blue}{\subsection{Is there a published document that describes an important error in this dataset (e.g., an erratum)? If so, please provide a link or other access point.}}
\textcolor{gray}{\textit{Motivation: Document any corrections to the dataset.}}

\textcolor{blue}{\subsection{Who is hosting the datasheet? Will the datasheet be updated in the future?}}
\textcolor{gray}{\textit{Motivation: Document stable access to the datasheet.}}

\textcolor{blue}{\subsection{Any other comments?}}
\textcolor{gray}{\textit{Motivation: Space for any other relevant information about data distribution and maintenance.}}
\vspace{10mm}

\section{Data-dependent questions}
Responses in this section will be dependent on the type(s) of data contained in the dataset. Questions that do not apply can be left blank.


\textcolor{blue}{\subsection{How was the data generated or collected? (e.g., a model used to produce output, reanalysis estimation of conditions, observations using remote sensing methods or in situ sensors) Please provide relevant citation(s); if none exist, describe why.}}
\textcolor{gray}{\textit{Motivation: Establish fundamental information about the methods used to generate or collect data in the dataset.}}

\textcolor{blue}{\subsection{If the data has been evaluated against some baseline(s) (e.g., an observational product or fundamental physical laws), please describe its evaluation against that baseline(s). If available, simply provide the relevant citation.}}
\textcolor{gray}{\textit{Motivation: Document adequacy of the method used to construct the data within the scope of this datasheet.}}

\textcolor{blue}{\subsection{Please note configurations or modifications made to any model used to complete runs in this dataset (e.g. changes to seasonality, changes to coupling, nudging), or provide relevant startup files.}}
\textcolor{gray}{\textit{Motivation: Be transparent about the exact setup of the model to create the data.}}

\textcolor{blue}{\subsection{Describe relevant uncertainties associated with this data or provide citation(s). If no formal analysis of uncertainties has been completed, then please state this here.}}
\textcolor{gray}{\textit{Motivation: Provide information about known uncertainties within the scope of the project.}}

\textcolor{blue}{\subsection{Did the method of generation or collection of the data change within the scope of the dataset?}}
\textcolor{gray}{\textit{Motivation: Be transparent about important changes to instruments or methodology within the dataset.}} % ?

\textcolor{blue}{\subsection{Are there any relevant unexplained but important numerical values (``magic numbers") that go into the generation, collection, or processing of this data? (e.g., model tuning values, calibration constants, machine learning hyperparameters)}}
\textcolor{gray}{\textit{Motivation: Define unique numerical values that exist within or impact this data, but may not be documented elsewhere.}}

\textcolor{blue}{\subsection{Is this dataset an ensemble? If so, how many members are there? Describe how the ensemble is perturbed, and whether there are relevant forms of variability that are not dispersed. Are there differences in coverage between the ensemble members?}}
\textcolor{gray}{\textit{Motivation: Describe the sampling, construction, and any important limitations of the ensemble.}}

\textcolor{blue}{\subsection{Are there relevant categories, groupings, or labels within the data? If so, how are these determined?}}
\textcolor{gray}{\textit{Motivation: Be transparent about the processes used to define groups within the data.}}

\textcolor{blue}{\subsection{Can users contribute to this dataset (e.g., citizen science or human labeling)? If so, please describe the process. Will these contributions be evaluated or verified? If so, please describe how. If not, why not?}}
\textcolor{gray}{\textit{Motivation: Describe if the data includes user contributions.}}

\textcolor{blue}{\subsection{Are there specific tasks for which the dataset should not be used? If so,
please provide a description.}}
\textcolor{gray}{\textit{Motivation: Address relevant gaps or inadequacies of the data for specific use cases.}}

\textcolor{blue}{\subsection{What are the direct or downstream impacts on humans from this dataset? The non-comprehensive checklist below is intended to prompt the reader to think of common impacts from data. Please check all that apply, and include a brief text description with stable links to any references. Additionally, please document potential impacts relevant to the scope of the dataset that are not included on the checklist.}}
\textcolor{gray}{\textit{Motivation: Reflect on the potential impacts (direct or downstream) of the dataset on human systems.}}

\newlist{todolist}{itemize}{2}
\setlist[todolist]{label=$\square$}

Direct
\begin{todolist}
% Check boxes by changing \item to \item[\checked]
  \item Does this dataset support reproducibility of a specific scientific finding or figure?
  \item Were there notable CO\textsubscript{2} emissions in creating this dataset? \\ 
  (e.g., from large machine learning models)
  \item Were there notable land use impacts from equipment? \\
  (e.g., in situ instruments during a field experiment)
  \item Was this dataset created through co-production of research? \\
  (e.g., for fieldwork in vulnerable communities)
  \item Does this dataset include identifying information? \\
  (e.g., community-level data, social information) 
  

\end{todolist}

Downstream
\begin{todolist}
  \item Is this dataset intended for development of a research tool? \\
  (e.g., model improvement, sensor design)
  \item Does this dataset support further use for novel research? (e.g., unrelated scientific studies)
  \item Would analysis of this dataset be policy relevant? \\
  (e.g., climate, environmental, public health issues)
  \item Would this dataset be considered actionable science? \\ 
  (e.g., completed with use by a specific stakeholder in mind)
  \item Could this dataset inspire behavioral changes? \\
  (e.g., change agricultural practices, city planning)
  \item Could this dataset affect operational forecasting? \\
  (e.g., improve models, forecasting, predictability)
\end{todolist}

\textcolor{blue}{\subsection{What biases were present in the construction or use of the dataset? The checklist below provides a non-exhaustive list of common examples in Earth science. Please check all that apply, and include a brief text description with stable links to any references. Additionally, please document any biases within the scope of the dataset that are not included in the checklist.}}
\textcolor{gray}{\textit{Motivation: Reflect on potential biases present in the dataset.}}

\begin{todolist}
    \item Geographic bias (e.g., restricted or weighted to specific regions)
    \item Model bias (e.g., error relative to observations or other ground truth product)
    \item Sensor bias (e.g., calibration)
    \item Day/night bias (e.g., diurnal cycle, restrictions)
    \item Seasonal biases (e.g., seasonal cycle)
    \item Bias towards extreme or standard conditions (e.g., catchment error in high winds, failure to represent extremes)
    \item Unbalanced sampling (e.g., unequal classes)
    \item Adversarial impacts on data (e.g., fraudulent data in crowdsourcing)
    \item Label bias (e.g., incorrect or incomplete labeling)
    \item Threshold sensitivity (e.g., for an extreme index)
    \item Regime dependence (e.g., convective structure, mode of variability)
\end{todolist}

\textcolor{blue}{\subsection{Any other comments? Are there any other citations necessary to document some important aspect of the data? If so, provide the citation(s) and describe their purpose.}}
\textcolor{gray}{\textit{Motivation: Space for any other relevant information about the data.}}


\bigskip
 
\bibliographystyle{unsrt}  
\bibliography{cite}
\end{document}
